\documentclass{article}

\usepackage{fancyhdr}
\usepackage{lastpage}
\usepackage{extramarks}
\usepackage[usenames,dvipsnames]{color}
\usepackage{courier}
\usepackage{amsmath}
\usepackage{amsthm}
\usepackage{amssymb}
\usepackage{amsfonts}
\usepackage{tikz}
\usepackage{enumerate}

\usetikzlibrary{automata,positioning}

\topmargin=-0.45in
\evensidemargin=0in
\oddsidemargin=0in
\textwidth=6.5in
\textheight=9.0in
\headsep=0.25in

\linespread{1.1}

\pagestyle{fancy}
\lhead{\hmwkAuthorName}
\chead{\hmwkClass\ (\hmwkClassInstructor\ \hmwkClassTime): \hmwkTitle}
\rhead{\firstxmark}
\lfoot{\lastxmark}
\cfoot{}
\renewcommand\headrulewidth{0.4pt}
\renewcommand\footrulewidth{0.4pt}

\setlength\parindent{0pt}

\newcommand{\enterProblemHeader}[1]{
\nobreak\extramarks{#1}{#1 continued on next page\ldots}\nobreak
\nobreak\extramarks{#1 (continued)}{#1 continued on next page\ldots}\nobreak
}

\newcommand{\exitProblemHeader}[1]{
\nobreak\extramarks{#1 (continued)}{#1 continued on next page\ldots}\nobreak
\nobreak\extramarks{#1}{}\nobreak
}

\setcounter{secnumdepth}{0}
\newcounter{homeworkProblemCounter}

\newcommand{\homeworkProblemName}{}
\newenvironment{homeworkProblem}[1][Problem \arabic{homeworkProblemCounter}]{
\stepcounter{homeworkProblemCounter}
\renewcommand{\homeworkProblemName}{#1}
\section{\homeworkProblemName}
\enterProblemHeader{\homeworkProblemName}
}{
\exitProblemHeader{\homeworkProblemName}
}

\newcommand{\problemAnswer}[1]{
\noindent\framebox[\columnwidth][c]{\begin{minipage}{0.98\columnwidth}#1\end{minipage}}
}

\newcommand{\homeworkSectionName}{}
\newenvironment{homeworkSection}[1]{
\renewcommand{\homeworkSectionName}{#1}
\subsection{\homeworkSectionName}
\enterProblemHeader{\homeworkProblemName\ [\homeworkSectionName]}
}{
\enterProblemHeader{\homeworkProblemName}
}

\newcommand{\hmwkTitle}{Homework\ \#1}
\newcommand{\hmwkDueDate}{September\ 5,\ 2014 at 05:00pm}
\newcommand{\hmwkClass}{CS311}
\newcommand{\hmwkClassTime}{12:40pm}
\newcommand{\hmwkClassInstructor}{Professor Lathrop}
\newcommand{\hmwkAuthorName}{Kien Nguyen}

\title{
\vspace{2in}
\textmd{\textbf{\hmwkClass:\ \hmwkTitle}}\\
\normalsize\vspace{0.1in}\small{Due\ on\ \hmwkDueDate}\\
\vspace{0.1in}\large{\textit{\hmwkClassInstructor\ \hmwkClassTime}}
\vspace{3in}
}

\author{\textbf{\hmwkAuthorName}}
\date{}

\begin{document}
\maketitle

\pagebreak

\begin{homeworkProblem}
Prove:
	\begin{enumerate}[a)]
		\item $12|\mathbb{N} \subseteq 3|\mathbb{N}$ \\
		For $n \in 12|\mathbb{N}, \exists k \in \mathbb{N}$ such that $n = 12k$ \\
		then $n = 3 * (4k)$ then $n \in 3|\mathbb{N}$ \\
		Therefore $12|\mathbb{N} \subseteq 3|\mathbb{N}$
		\item $35|\mathbb{N} = 5|\mathbb{N} \cap 7|\mathbb{N}$ 
		\begin{enumerate}[(1)]
		\item we prove $35|\mathbb{N} \subseteq 5|\mathbb{N} \cap 7|\mathbb{N}$ \\
		For $n \in 12|\mathbb{N}, \exists k \in \mathbb{N}$ such that $n = 35k$ \\
		Hence, we have $n = 5 * (7k)$ and $n = 7 * (5k)$ then 
		$n \in 5|\mathbb{N}$ and $n \in 7|\mathbb{N}$.\\
		So $n \in 5|\mathbb{N} \cap 7|\mathbb{N}$ (done for-arrow)
		\item we prove $5|\mathbb{N} \cap 7|\mathbb{N} \subseteq 35|\mathbb{N}$\\
		For $n \in 5|\mathbb{N} \cap 7|\mathbb{N}$, we can have $n \in 5|\mathbb{N}$ 
		and $n \in 7|\mathbb{N}$.
		Since $n \in 5|\mathbb{N} \exists k \in \mathbb{N}$ such that $n = 5k$\\
		Then $5k \in 7|\mathbb{N}$ then $5k$ is divisible by $7$\\
		Because $(5, 7) = 1$ (largest common divisor), then $k$ is divisible by $7$.\\
		So $\exists h \in \mathbb{N}$ such that $k = 7h$. Then $n = 5 * (7h)$
		or $n = 35h$. Then $n \in 35|\mathbb{N}$ (done reversed-arrow).\\
		Therefore, $12|\mathbb{N} \subseteq 3|\mathbb{N}$.
		\end{enumerate}
		\item $20|\mathbb{N} \nsubseteq 3|\mathbb{N}$\\
		In order for $20|\mathbb{N} \subseteq 3|\mathbb{N}$, every single element in $20|\mathbb{N}
		$ must be in $3|\mathbb{N}$.\\
		Since $20 \in 20|\mathbb{N}$, but $20 \notin 3|\mathbb{N} \Rightarrow 20|\mathbb{N} \nsubseteq 
		3|\mathbb{N}$
	\end{enumerate}
\end{homeworkProblem}
\pagebreak
\begin{homeworkProblem}
For arbitrary sets A,B, prove:
\begin{enumerate}[a)]
\item $A \cup B = B \iff A \subseteq B$
\item $A \cap B = B \iff B \subseteq A$
\item $A - (A - B) \subseteq B$\\
And prove there exists sets $A, B$ such that:
\item $B \nsubseteq A - (A - B)$
\end{enumerate}
\begin{proof}
\begin{enumerate}[(a)]
\item $(\Rightarrow)$ Prove: If $A \cup B = B \Rightarrow A \subseteq B$\\
For $\forall x \in A$, $x \in A \cup B$. But we also have $A \cup B = B \Rightarrow x \in B$.\\
So $\forall x \in A$, also we have $x \in B$.\\
Therefore, $A \subseteq B$. (done forward arrow)\\
$(\Leftarrow)$ Prove: If $A \subseteq B \Rightarrow A \cup B = B$\\
By definition, $A \cup B = \{x| x \in A \parallel x \in B\}$.\\
We also have for every $x \in A$, $x \in B$ also (because $A \subseteq B$.)\\
$\Rightarrow A \cup B = \{x| x \in B\} = B$. (done reversed arrow)
\item $(\Rightarrow)$ Prove: If $A \cap B = B \Rightarrow B \subseteq A$\\
For $\forall x \in B$, we also have $x \in A \cap B$ because $B = A \cap B$.
Therefore, also we have $x \in A$, $\Rightarrow B \subseteq A$. (done forward arrow)\\
$(\Leftarrow)$ Prove: If $B \subseteq A \Rightarrow A \cap B = B$\\
By definition, $A \cap B = \{x|x \in A \wedge x \in B$.\\
Because $B \subseteq A \Rightarrow \forall x \in B$, then apparently $x \in A$.\\
Therefore, $A \cap B = \{x|x \in B\} = B$. (done reversed arrow)
\item Using venn diagrams
\item $A = \{0,1\}, B = \{1,2\} \Rightarrow A - (A - B) = \{1\}$.\\
Therefore, $B \nsubseteq A - (A - B)$.
\end{enumerate}
\end{proof}
\end{homeworkProblem}

\pagebreak

\begin{homeworkProblem}
Give an example of a function $f: \mathbb{Z} \rightarrow \mathbb{N}$ that is both one-to-one and onto
\begin{proof}
Choose $f$ such that: 
\[ f(n) = \left\{
	\begin{array}{l l}
		2n + 1 & \quad \text{if $n \geq 0$}\\
		2|n| & \quad \text{if $n < 0$} 
	\end{array}\right.\]
\begin{enumerate}[(a)]
\item Prove f is one-to-one\\
For any number $n,m \in \mathbb{Z}$, we have $f(n) = f(m) \iff n, m \geq 0 \parallel n, m < 0$\\
If $n,m \geq 0$, then $f(n) = f(m) \iff 2n + 1 = 2m + 1 \iff n = m$.\\
If $n,m <0$, then $f(n) = f(m) \iff 2|n| = 2|m \iff n = m$.\\
Therefore $f(n) = f(m) \iff n = m \Rightarrow f$ is one-to-one.
\item Prove f is onto\\
For any number $v \in \mathbb{N}$, \\
If $v$ is odd, we can choose $n = (v-1)/2$ such that $f(n) = v$.\\
If $v$ is even, we can choose $n = -v/2$ such that $f(n) = v$.\\
Therefore f is onto.
\end{enumerate}
\end{proof}
\end{homeworkProblem}

\pagebreak

\begin{homeworkProblem}
Let $f: \mathbb{Z} \rightarrow \mathbb{Z}$ be a function defined as $f(n) = 3x + 7$.Prove:
\begin{enumerate}[(a)]
\item $f$ is one to one
\item $f$ is NOT onto
\end{enumerate}
\begin{proof}
\begin{enumerate}[(a)]
\item For every $x, y \in \mathbb{Z}$, we have: $f(x) = f(y) \Leftrightarrow 3x + 7 = 3y + 7
\Leftrightarrow x = y \Rightarrow f$ is one to one. 
\item Let $v = 8$. Assume that there is a value $x \in Z$ such that $f(x) = v$ or $f(x) = 8$\\
$\Rightarrow 3x + 7 = 8 \Rightarrow x = 1/7 \Rightarrow x \notin \mathbb{Z} \Rightarrow$ contradiction.\\
Therefore $f$ is NOT onto.
\end{enumerate}
\end{proof}
\end{homeworkProblem}

\pagebreak

\begin{homeworkProblem}
Let $r$ be a relation over real numbers such that for $a,b \in \mathbb{R}, a$ $r$ $b$ if and only if
$a - b \in \mathbb{Z}$. Prove that $r$ is an equivalence relation.
\begin{proof}
\begin{enumerate}[(1)]
\item Prove $r$ is reflective\\
For $\forall x \in \mathbb{R}$, $x - x = 0 \in \mathbb{Z} \Rightarrow r$ is reflective.
\item Prove $r$ is symmetric\\
For $x,y \in \mathbb{R}$, assume that $x - y \in \mathbb{Z} \Rightarrow x - y = z \in \mathbb{Z}$
$\Rightarrow y - x = -z \in \mathbb{Z} \Rightarrow r$ is symmetric.
\item Prove $r$ is transitive\\
For $x,y,z \in \mathbb{R}$, assume that $x - y = a \in \mathbb{Z}$ and $y - z = b \in \mathbb{Z}$\\
$\Rightarrow x - z = (x-y) + (y-z) = a + b \in \mathbb{Z} \Rightarrow r$ is transitive.
\end{enumerate}
So, $r$ is an equivalence relation.
\end{proof}
\end{homeworkProblem}
\pagebreak

\begin{homeworkProblem}
Use induction to prove the following:
\begin{enumerate}[a)]
\item For all $n \in \mathbb{Z}^+$,
\[ 1 + 3 + 5 + \dots + 2n - 1 = n^2\]
Base case: $n = 1 \Rightarrow 1 + 3 + 5 + \dots + 2n - 1 = 1 = n^2 \Rightarrow$ statement is true for 
$n = 1$.\\
Induction: Assume the statement is true for $n$, that is $1 + 3 + 5 + \dots + 2n - 1 = n^2$, we'll
prove that the statement also holds for $n+1$, that is 
$1 + 3 + 5 + \dots + 2(n+1) - 1 = (n+1)^2$.\\
$1 + 3 + 5 + \dots + 2(n+1) - 1 = 1+3+5 + \dots + 2n - 1 + 2(n + 1) - 1
= n^2 + 2(n+1) - 1 = n^2 + 2n + 1 = (n+1)^2$. (done)
\item For all $n \in \mathbb{Z}^+$,
\[3^n > 2^n\]
Base case: $n = 1 \Rightarrow 3^n = 3$ and $2^n = 2 \Rightarrow 3^n > 2^n \Rightarrow$ statement is true for $n = 1$.\\
Induction: Assume the statement is true for $n$, that is $3^n > 2^n$, we'll prove that the statement is also holds for $n+1$, that is $3^{n+1} > 2^{n+1}$.\\
We have $3^{n+1} = 3^n3$ and $2^{n+1} = 2^n2$.\\
Since $3^n > 2^n$ and $3 > 2$, then $3^n3 > 2^n2 \Rightarrow 3^{n+1} > 2^{n+1}$. (done)
\item For all $n \in \mathbb{Z}^+$,
\[\sum\limits_{i=1}^n i = \frac{n(n+1)}{2}\]
Base case: $n=1$, $\sum\limits_{i=1}^n i = 1 = \frac{n(n+1)}{2} \Rightarrow$ statement is true for $n = 1$.\\
Induction: Assume the statement is true for $n$, that is $\sum\limits_{i=1}^n i = \frac{n(n+1)}{2}$,
we'll prove that the statement is also holds for $n+1$, that is
$\sum\limits_{i=1}^{n+1} i = \frac{(n+1)(n+2)}{2}$
\[\sum\limits_{i=1}^{n+1} i = \sum\limits_{i=1}^{n} i + (n+1) = \frac{n(n+1)}{2} + (n+1)\]
\[= \frac{n^2 + n + 2n + 2}{2} = \frac{(n+1)(n+2)}{2}\]
(done)
\item For all $n \in \mathbb{Z}^+$,
\[n^3 + 2n \text{ is divisible by 3}\]
Base case: $n=1$,$n^3 + 2n = 3$ is divisible by 3. So, statement is true for $n = 1$.\\
Induction: Assume the statement is true for $n$, that is $n^3 + 2n = 3k$, where $k \in \mathbb{Z}^+$, 
we'll prove that the statement is also holds for $n+1$, that is 
$(n+1)^3 + 2(n+1)$ is also divisible by 3.\\
$(n+1)^3 + 2(n+1) = n^3 + 3n^2 + 3n + 1 + 2n + 2 = (n^3 + 2n) + (3n^2 + 3n + 3) =
3k + 3(n^2 + n + 1) = 3(k + n^2 + n + 1)$ is divisible by 3. (done)  
\end{enumerate}
\end{homeworkProblem}
\end{document}
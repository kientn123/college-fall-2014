\documentclass{article}

\usepackage{fancyhdr}
\usepackage{lastpage}
\usepackage{extramarks}
\usepackage[usenames,dvipsnames]{color}
\usepackage{courier}
\usepackage{amsmath}
\usepackage{amsthm}
\usepackage{amsfonts}
\usepackage{tikz}
\usepackage{enumerate}

\usetikzlibrary{automata,positioning}

\topmargin=-0.45in
\evensidemargin=0in
\oddsidemargin=0in
\textwidth=6.5in
\textheight=9.0in
\headsep=0.25in

\linespread{1.1}

\pagestyle{fancy}
\lhead{\hmwkAuthorName}
\chead{\hmwkClass\ (\hmwkClassInstructor\ \hmwkClassTime): \hmwkTitle}
\rhead{\firstxmark}
\lfoot{\lastxmark}
\cfoot{}
\renewcommand\headrulewidth{0.4pt}
\renewcommand\footrulewidth{0.4pt}

\setlength\parindent{0pt}

\newcommand{\enterProblemHeader}[1]{
\nobreak\extramarks{#1}{#1 continued on next page\ldots}\nobreak
\nobreak\extramarks{#1 (continued)}{#1 continued on next page\ldots}\nobreak
}

\newcommand{\exitProblemHeader}[1]{
\nobreak\extramarks{#1 (continued)}{#1 continued on next page\ldots}\nobreak
\nobreak\extramarks{#1}{}\nobreak
}

\setcounter{secnumdepth}{0}
\newcounter{homeworkProblemCounter}

\newcommand{\homeworkProblemName}{}
\newenvironment{homeworkProblem}[1][Problem \arabic{homeworkProblemCounter}]{
\stepcounter{homeworkProblemCounter}
\renewcommand{\homeworkProblemName}{#1}
\section{\homeworkProblemName}
\enterProblemHeader{\homeworkProblemName}
}{
\exitProblemHeader{\homeworkProblemName}
}

\newcommand{\problemAnswer}[1]{
\noindent\framebox[\columnwidth][c]{\begin{minipage}{0.98\columnwidth}#1\end{minipage}}
}

\newcommand{\homeworkSectionName}{}
\newenvironment{homeworkSection}[1]{
\renewcommand{\homeworkSectionName}{#1}
\subsection{\homeworkSectionName}
\enterProblemHeader{\homeworkProblemName\ [\homeworkSectionName]}
}{
\enterProblemHeader{\homeworkProblemName}
}

\newcommand{\hmwkTitle}{Homework\ \#1}
\newcommand{\hmwkDueDate}{September\ 5,\ 2014 at 09:00am}
\newcommand{\hmwkClass}{CS311}
\newcommand{\hmwkClassTime}{09:00pm}
\newcommand{\hmwkClassInstructor}{Professor Lutz}
\newcommand{\hmwkAuthorName}{Kien Nguyen}

\title{
\vspace{2in}
\textmd{\textbf{\hmwkClass:\ \hmwkTitle}}\\
\normalsize\vspace{0.1in}\small{Due\ on\ \hmwkDueDate}\\
\vspace{0.1in}\large{\textit{\hmwkClassInstructor\ \hmwkClassTime}}
\vspace{3in}
}

\author{\textbf{\hmwkAuthorName}}
\date{}

\begin{document}
\maketitle

\pagebreak

\begin{homeworkProblem}
Prove or disapprove: If $A = {0^n1^n|n \in \mathbb{N}}$, then $A^* = A$ 
\begin{proof}
We'll disapprove it. \\
Assume that $A = {0^n1^n|n \in \mathbb{N}}$, we have $0101 \in A^*$, but $0101 \notin A$. Therefore,
$A^* \neq A$
\end{proof}
\end{homeworkProblem}

\pagebreak

\begin{homeworkProblem}
Prove or disapprove: If $B = \{x \in \{0,1\}^*|\#(0,x) = \#(1,x)\}$, then $B* = B.$\\
Note: The notation $\#(0,x)$ is used to denote the number of $0's$ in $x$. Likewise, $\#(1,x)$ is used
to denote the number of $1's$ in $x$.
\begin{proof}
\begin{enumerate}[(1)]
\item Prove $B* \in B$. \\
For every $y \in B^*$, we have $y$ is a combination of some elements in $B$. Therefore we can write: \\
\[ y = x_1x_2 \dots x_n, \, where \; n \in \mathbb{N}, \, and \, x_i \in B, i=1 \dots n \]
then,\\
\[ \#(0,y) = \#(0,x_1) + \#(0,x_2) + \dots + \#(0,x_n) \] and
\[ \#(1,y) = \#(1,x_1) + \#(0,x_2) + \dots + \#(1,x_n) \]
Since we have $x_i \in B$ then $\#(0,x_i) = \#(1,x_i) \forall i=1 \dots n$, then $\#(0,y) = \#(1,y)$
$\Rightarrow y \in B$. Therefore $B^* \subseteq B$.
\item Prove $B \in B^*$. \\
For any alphabet $A$, each element in $A$ can itself represent a string in $A^*$, therefore $A \in A^*$
for every $A$.\\
So, we have $B \in B^*$.
\end{enumerate}
From (1) and (2), we have $B = B^*$.
\end{proof}

\end{homeworkProblem}

\pagebreak
\begin{homeworkProblem}
Prove: For every positive integer $n$, 
\[ \sum\limits_{i=1}^n \frac{1}{k^2} \leq 2 - \frac{1}{n} \]
\begin{proof}
We'll prove this problem by induction on n.\\
\begin{enumerate}[(1)]
\item Base case: $n = 1$, we need \[\frac{1}{1^1} \leq 2 - \frac{1}{1}\]and that is true.
\item We assume that the statement is true for n, that is $\sum\limits_{i=1}^n \frac{1}{k^2} \leq 2 - \frac{1}{n}$, we will prove that the statement also holds for n+1, that is $\sum\limits_{i=1}^{n+1} \frac{1}{k^2} \leq 2 - \frac{1}{n+1}$.\\
We have 
\[ \sum\limits_{i=1}^{n+1} \frac{1}{k^2} = \sum\limits_{i=1}^{n} \frac{1}{k^2} + \frac{1}{(n+1)^2}
\leq 2 - \frac{1}{n} + \frac{1}{(n+1)^2} \]
\[ = 2 - \frac{n^2 + n + 1}{n(n+1)^2} = 2 - \frac{n(n+1) + 1}{n(n+1)^2} \]
\[ \leq 2 - \frac{n(n+1)}{n(n+1)^2} = 2 - \frac{1}{n+1} \]
So, 
\[ \sum\limits_{i=1}^{n+1} \frac{1}{k^2} \leq 2 - \frac{1}{n+1} \]

\end{enumerate}
\end{proof}
\end{homeworkProblem}

\pagebreak

\begin{homeworkProblem}
Prove: For every language $A, A^{**} = A^*$.
\begin{proof}
\begin{enumerate}[(1)]
\item Since each element in $A^*$ can itself represent a string in $A^{**}$, then $A^* \subseteq 
A^{**}$.
\item We'll prove that $A^{**} \subseteq A^*$. \\
Each element $y \in A^{**}$ can be represented as $y = x_1x_2 \dots x_n$ where $n \in \mathbb{N}$
and $x_i \in A^*$ for $i = 1 \dots n$.\\
Because $x_i \in A^*$, $x_i$ is a concatenation from elements from $A \Rightarrow x_1x_2 \dots 
x_n$ is also a concatenation from elements from $A \Rightarrow y$ is also a concatenation from 
elements from $A \Rightarrow y \in A^*$. \\
Therefore, $A^{**} \subseteq A^*$.
\end{enumerate}
From (1) and (2), we have $A^{**} = A^*$.
\end{proof}
\end{homeworkProblem}

\pagebreak

\begin{homeworkProblem}
Prove: If $S = \{0,1\}$ and $T \subseteq \{0,1\}^*$, then 
\[ S^* = T^* \Rightarrow S \subseteq T \]
\begin{proof}
Since $S = \{0,1\}$, $0 \in S^*$ and $1 \in S^*$. Followed by the fact that $S^* = T^* \Rightarrow 
0 \in T^*$ and $1 \in T^*$. \\
$T \subseteq \{0,1\}^* \Rightarrow T$ is a list of strings where each string is a concatenation
of $0's$ and $1's$. Since $T^*$ contains $0$ and $1$, and $0$ and $1$ are single characters, then there
is no way that elements in $T$ can produce a single $0$ and a single $1$ in $T^*$ without $T$ actually 
contains $0$ and $1$ itself $\Rightarrow 0 \in T$ and $1 \in T$.\\
So, all the elements in $S$ are in $T \Rightarrow S \subseteq T$.
\end{proof}
\end{homeworkProblem}

\pagebreak

\begin{homeworkProblem}
Exhibit languages $S,T \subseteq \{0,1\}^*$ such that $S^* = T^*$ and $\{0,1\} \subseteq S \subset 
T$.
\begin{proof}
We pick $S = \{0,1,01\}$ and $T = \{0,1,01,10\}$. Both $S$ and $T$ contain $0$ and $1 \Rightarrow 
\{0,1\} \subseteq S$ and $\{0,1\} \subseteq T$.\\
We can also see that $S \subset T$.\\
We'll prove that $S^* = T^*$.  
\begin{enumerate}[(1)]
\item Prove $S^* \subseteq T^*$ \\
For $y \in S^*$, $y$ is sure a concatenation of $0$ and $1$. And we have $0 \in T$ and $1 \in T$.\\
Therefore $y \in T^* \Rightarrow S^* \subseteq T^*$.
\item Similarly, we also have $T^* \subseteq S^*$.
\end{enumerate}
So, $S^* = T^*$.
\end{proof}
\end{homeworkProblem}

\pagebreak

\begin{homeworkProblem}
Define an (infinite) binary sequence $s \in \{0,1\}^\infty$ to be $prefix-repetitive$ if there are
infinitely many strings $w \in \{0,1\}^*$ such that $ww \sqsubseteq s$.\\
Prove: If the bits of a strings $s \in \{0,1\}^\infty$ are chosen by independent tosses of a fair coin,
then
\[ Prob[s\;is\;prefix-repetitive] = 0. \]
Note: $x \sqsubseteq y$ means that $x$ is a prefix of $y$ where $x$ and $y$ are strings.
\begin{proof}
Because if $s$ is prefix-repetitive, there are infinitely many $w \in \{0,1\}^*$ such that $ww 
\sqsubseteq s$, then we can have the length of $w$ can grow to $\infty$.\\
Assume that after $n$ tosses, we have a string $w$, the probabilty such that after another $n$ tosses,
we have the whole string as $ww$ is $(\frac{1}{2})^n$ (since it is a fair coin $\Rightarrow$ the
probability to have head or tail is 1/2, and since we want the second n tosses to be exactly the same
as the first n tosses $\Rightarrow$ each toss of the second n tosses must be the same as the correspond
toss from the first n tosses $\Rightarrow$ for each one, we have only 1/2 chance it happens).\\
Because $n$ can grow to $\infty$, $(\frac{1}{2})^n$ can grow to $0$.\\
$\Rightarrow$ Prob[having $ww$ in the string] is 0
$\Rightarrow$ Prob[s is prefix-repetitive] is 0.
\end{proof}
\end{homeworkProblem}

\pagebreak

\begin{homeworkProblem}
Define $2$-$coloring$ of $\{0,1\}^*$ to be a function $\mathcal{X} : \{0,1\}^* \rightarrow \{red, blue
\}$. (For examle, if $\mathcal{X}(1101)$ = red, we say that 1101 is red in coloring $\mathcal{X}$.)\\
Prove: For every 2-coloring $\mathcal{X}$ and every (infinite) binary sequence $s \in \{0,1\}^\infty$,
there is a sequence
\[w_0,w_1,w_2,...\]
of strings $w_n \in \{0,1\}^*$ such that
\begin{enumerate}[(i)]
\item $s = w_0w_1w_2\dots$, and
\item $w_1, w_2, w_3, \dots$ are all the same color. (The string $w_0$ may not be this color.)
\end{enumerate}
\end{homeworkProblem}
\begin{proof}
I don't know how to solve this problem
\end{proof}
\end{document}